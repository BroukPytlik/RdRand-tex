%============================================================================
% tento soubor pouzijte jako zaklad
% (c) 2008 Michal Bidlo
% E-mail: bidlom AT fit vutbr cz
%============================================================================
% kodovaní: iso-8859-2 (zmena prikazem iconv, recode nebo cstocs)
%----------------------------------------------------------------------------
% zpracování: make, make pdf, make desky, make clean
% připomínky posílejte na e-mail: bidlom AT fit.vutbr.cz
% vim: set syntax=tex encoding=latin2:
%============================================================================
\documentclass[english]{jtulak} % odevzdani do wisu - odkazy, na ktere se da klikat
%\documentclass[cover,print]{fitthesis} % pro tisk - na odkazy se neda klikat
%\documentclass[english,print]{fitthesis} % pro tisk - na odkazy se neda klikat
%      \documentclass[english]{fitthesis}
% * Je-li prace psana v anglickem jazyce, je zapotrebi u tridy pouzit 
%   parametr english nasledovne:
%      \documentclass[english]{fitthesis}
% * Neprejete-li si vysazet na prvni strane dokumentu desky, zruste 
%   parametr cover

% zde zvolime kodovani, ve kterem je napsan text prace
% "latin2" pro iso8859-2 nebo "cp1250" pro windows-1250, "utf8" pro "utf-8"
%\usepackage{ucs}
\usepackage[utf8]{inputenc}
%\usepackage[T1, IL2]{fontenc}
\usepackage{url}
\DeclareUrlCommand\url{\def\UrlLeft{<}\def\UrlRight{>} \urlstyle{tt}}

%zde muzeme vlozit vlastni balicky

\usepackage{sidecap}
\usepackage{listings}
\usepackage{syntonly}
%\syntaxonly


% =======================================================================
% balíček "hyperref" vytváří klikací odkazy v pdf, pokud tedy použijeme pdflatex
% problém je, že balíček hyperref musí být uveden jako poslední, takže nemůže
% být v šabloně

\ifx\pdfoutput\undefined % nejedeme pod pdflatexem
\else
  \ifNotPrint
    \usepackage{color}
    \usepackage[unicode,colorlinks,hyperindex,plainpages=false,pdftex]{hyperref}
    \definecolor{links}{rgb}{0.4,0.5,0}
    \definecolor{anchors}{rgb}{1,0,0}
    \def\AnchorColor{anchors}
    \def\LinkColor{links}
    \def\pdfBorderAttrs{/Border [0 0 0] }  % bez okrajů kolem odkazů
    \pdfcompresslevel=9
  \fi
\fi

%Informace o praci/projektu
%---------------------------------------------------------------------------
\projectinfo{
  %Prace
  project=BP,            %typ prace BP/SP/DP/DR
  year=2014,             %rok
  date=\today,           %datum odevzdani
  %Nazev prace
  title.cs={RdRand: IA-64 a IA-32 instrukce pro generování náhodných čísel},  %nazev prace v cestine
  title.en={RdRand: IA-64 and IA-32 Instruction for Random Number Generation}, %nazev prace v anglictine
  %Autor
  author={Jan Ťulák},   %jmeno prijmeni autora
  %author.title.p=Bc., %titul pred jmenem (nepovinne)
  %author.title.a=PhD, %titul za jmenem (nepovinne)
  %Ustav
  department=UPSY, % doplnte prislusnou zkratku: UPSY/UIFS/UITS/UPGM
  %Skolitel
  supervisor= Tomáš Kašpárek, %jmeno prijmeni skolitele
  supervisor.title.p=Ing.,   %titul pred jmenem (nepovinne)
  supervisor.title.a={},    %titul za jmenem (nepovinne)
  %Klicova slova, abstrakty, prohlaseni a podekovani je mozne definovat 
  %bud pomoci nasledujicich parametru nebo pomoci vyhrazenych maker (viz dale)
  %===========================================================================
  %Klicova slova
  keywords.cs={RdRand, Intel, náhodná čísla, knihovna, generátor}, %klicova slova v ceskem jazyce
  keywords.en={RdRand, Intel, random numbers, library, generator}, %klicova slova v anglickem jazyce
  %Abstract
  abstract.cs={Cílem této bakalářské práce bylo vytvoření linuxové knihovny pro usnadnění práce s instrukcí RdRand na procesorech Intel v aplikacích třetích stran, podle požadavků společnosti Red Hat. V práci jsou rovněž poskynuty a vyhodnoceny výsledky z výkonnostních a statistických testů.}, % abstrakt v ceskem jazyce
  abstract.en={This thesis aimed to creation of a Linux library for easier usage of RdRand instruction on Intel CPUs in third-party applications according requirements of Red Hat. Also, results of performance and statistical tests are included. }, % abstrakt v anglickem jazyce
  %Prohlaseni
  declaration={Hereby I declare that I wrote this work on my own and all used sources are stated and correctly noted as citations.},
  %Podekovani (nepovinne)
  acknowledgment={Zde je možné uvést poděkování vedoucímu práce a těm, kteří poskytli odbornou pomoc.} % nepovinne
}

%Abstrakt (cesky, anglicky)
%\abstract[cs]{Do tohoto odstavce bude zapsán výtah (abstrakt) práce v českém jazyce.}
%\abstract[en]{Do tohoto odstavce bude zapsán výtah (abstrakt) práce v anglickém jazyce.}

%Klicova slova (cesky, anglicky)
%\keywords[cs]{Sem budou zapsána jednotlivá klíčová slova v českém jazyce, oddělená čárkami.}
%\keywords[en]{Sem budou zapsána jednotlivá klíčová slova v anglickém jazyce, oddělená čárkami.}

%Prohlaseni
%\declaration{Prohlašuji, že jsem tuto bakalářskou práci vypracoval samostatně pod vedením pana X...
%Další informace mi poskytli...
%Uvedl jsem všechny literární prameny a publikace, ze kterých jsem čerpal.}

%Podekovani (nepovinne)
%\acknowledgment{V této sekci je možno uvést poděkování vedoucímu práce a těm, kteří poskytli odbornou pomoc
%(externí zadavatel, konzultant, apod.).}

%%%%%%%%%%%%%%%%%%%%%%%%%

% my macro for function declaration
\newcommand{\FunctionDeclareNE}[4]{%
  \noindent
  \phantomsection
  \label{fnt:#2}
  \textbf{{\em #1} {\tt #2} ( {\em #3} ) }-- {#4}
  \endgroup}
% because underscore is a special character and has to be escaped
% but escaped sequences must not be in \label
\def\FunctionDeclare{\begingroup 
\catcode`\_=12
\FunctionDeclareNE}

% my macro for linking to a function declaration
\newcommand{\functionNE}[1]{%
  \hyperref[fnt:#1]{\tt #1}
  \endgroup}
% because underscore is a special character and has to be escaped
% but escaped sequences must not be in \label
\def\function{\begingroup 
\catcode`\_=12
\functionNE}

%%%%%%%%%%%%%%%%%%%%%
% my macro for function declaration
\newcommand{\MachineDeclareNE}[3]{%
  \noindent
  \phantomsection
  \label{machines:#1}
  \textbf{#1}\\CPU: {\it #2}\\Notes: {\it #3}\\
  \endgroup}
% because underscore is a special character and has to be escaped
% but escaped sequences must not be in \label
\def\machineDeclare{\begingroup 
\catcode`\_=12
\MachineDeclareNE}

\newcommand{\machineNE}[1]{%
  \hyperref[machines:#1]{\it #1}
  \endgroup}
% because underscore is a special character and has to be escaped
% but escaped sequences must not be in \label
\def\machine{\begingroup 
\catcode`\_=12
\machineNE}


%%%%%%%%%%%%%%%%%%%%%

\begin{document}

\pagenumbering{Alph}
  % Vysazeni titulnich stran
  % ----------------------------------------------
  \maketitle
  % Obsah
  % ----------------------------------------------
\pagenumbering{arabic}
  \tableofcontents
  
  % Seznam obrazku a tabulek (pokud prace obsahuje velke mnozstvi obrazku, tak se to hodi)
  % \listoffigures
  % \listoftables 

  % Text prace
  % ----------------------------------------------
  %=========================================================================
% (c) Jan Tulak (2013)

%=========================================================================
\chapter{Introduction}
Generating of true random numbers is a stochastic process. In opposite, computers are deterministic machines and as such, they are unable to~generate true random numbers just using abilities of a~Turing machine. \TODO{quote it} %TODO quote it
But random numbers are crucial in cryptography and once computers began to be used in this domain, \TODO{more details} %TODO more details
people tried combine these two conflicting requests - to~allow a~deterministic machine to act stochasticaly. 

There are two solutions of this problem. We can stay with completely deterministic machines and through series of mathematical operations compute pseudo random numbers, that seems to be random, but when using the same initial state and algorithm, we get the same numbers again. Or we can add some source of entropy to the system, a device that is physically stochastic and measure its output (wherever it be temperature, radiation, or anything else what can be later, directly or indirectly, converted to electric current). The second approach is providing real random numbers, but frequently with much lower speed, than pseudo-random algorithms. Also, finding an entropy source with a good speed, an~uniform distribution and with a reliable price, size, energy consumption and other parameters is difficult, so frequently the generated random values are used as a~seed for a~pseudo-random algorithm, which can lead to great speed enhancement without loosing much of the~randomness.

The problematic of~quality of the entropy for a~specific purpose is wide and this work is not intended to cover it to great length, but still it has be shortly mentioned. Clearly, different requests has an developer of a video game, a researcher performing a simulation and an~army for encrypting their information. The researcher needs numbers that seems to be random, but aren't - he or she needs to be able to repeat the simulation with the same initial state to get the same result\footnote{In some cases of computing, researchers are even keeping the same machines, as a different machine would provide a different results due to inner number representation and such.\cite{ArithmeticInCloud}\TODO{Is it ok?}%TODO is it ok?
}
The video game developer can also need the repeatibility (e.g. for generating a terrain), but in another case, like decisions of a artificial intelligence, the repeatibility may not be requested and in case of gambling highly unwanted. And the~army needs the random numbers generator to be completely stochastic to prevent an enemy to decipher their messages. Each of these cases has different requests for quality, speed and price.

Due to prices of {\em Hardware Random Number Generators (HW RNG)}, they were for long time domain of just governments and big corporations,\TODO{Quote it}%TODO quote it
leaving the consumer electronic to rely only on {\em Pseudo-Random Number Generators (PRNG)}. PRNG algorithms developed to great success over time, providing enough entropy for usual needs of ordinary people and also for most of cryptographic needs ({\em Cryptographically Secure Random Number Generators - CSPRNG}), but as more and more of our daily life depends on computers, the importance of keeping our data secured raised up. 

In 2012\cite{IntelRdRandFindAbout} Intel added a Digital Random Number Generator (DRNG) on their chips and allowed programmers to use it as part of instruction set of that CPUs. Intel named the instruction RdRand and its own implementation and the underlying DRNG hardware implementation is named {\em Intel Secure Key} (previously code-named Bull Mountain Technology)\cite{IntelDRNGAnalysis}. Intel used combination of HW RNG and CSPRNG, solution known as {\em Cascade Construction RNG}, where the relatively slow HW RNG works as a seed generator for much faster CSPRNG. As is showed later in this thesis, in chapter~\nameref{chap:performance}, the difference in speed is about thousand times.

This step brought HW RNGs to general public, but the limitation on just Intel CPUs means that there is still big part of the information technology market without such solution - in x86 world there is also AMD who did not yet implemented this instruction and also many of different architectures, like ARM, in mobile devices. So the programmers cannot count on the presence of a HW RNG on a casual computers yet. Furthermore, the Intel's RdRand instruction is still mostly unknown and there are also a questions about reliability of this generator, which is sealed into the chip and unable to be audited.\cite{TheodoreTsoNSA}

As I'm interested in computer security on some level, as well as in Linux, when I got the possibility to work on implementation of a library using RdRand, as well as test it, in production environment of Red Hat corporation, I was interested in it and choose it as my thesis. During the work, we have discovered unexpected issue with performance on some CPUs which was handed to Intel, yet until now without a response.

As now, in 2013, no other implementation of RdRand than Intel's one exists\footnote{Although AMD is working on their own implementation.\TODO{Find more about it }},%TODO find more about it - http://extrahardware.cnews.cz/cpu-architektury-excavator-budou-umet-avx2-rdrand-dalsi-nove-instrukce
the term {\em RdRand} is used just as the instruction implemented by {\em Intel Secure Key} technology.

\TODO{Update it when news come.}%TODO Update it when news come.

\chapter{Random numbers and deterministic machines}
How to generate them?
\section{Pseudorandom numbers}
How pseudorandom numbers can be generated.

\section{Real random numbers}
We need some entropy source - user input, network communication... and pass it as a seed to pseudorandom generator.
%========================================================================= 


%=========================================================================
\chapter{The RdRand instruction}  \label{chap:rdrand-instruction}
First public information about RdRand came somewhen during year 2011\cite{IntelRdRandFindAbout}, a year before the CPUs with it were released and Intel itself send patches to add support into Linux in summer of the same year\cite{KernelRdRand}. Later, RdRand was added between Linux entropy sources for {\tt /dev/[u]random}.\TODO{Find discussion}% TODO find discussion
According to known information\cite{TheodoreTsoNSA}, Intel tried to have {\tt /dev/[u]random} rely only on their instruction, but that was denied. 

We weren't able to find any relevant data about usage of RdRand in Windows kernel, probably because all such negotiations happened behind closed doors. In user space, there is no difference between Linux and Windows; when it is possible to call the instruction, user space applications can use it. 

After disclosure of extends of NSA spying activities by Edward Snowden in summer of 2013\cite{GuardianNSA}\cite{DailymailNSA}, a petition for removing RdRand from Linux entropy sources was created\cite{PetitionRdRand}. Although supported by just 8 signatures, it got wide attention on information-technology aimed news pages and magazines, like Slashdot.org\cite{PetitionRdRandSlashdot}. The petition was closed after Linus Torvalds responds with scorn:

\begin{quote} ...

Short answer: we actually know what we are doing. You don't.

Long answer: we use rdrand as \_one\_ of many inputs into the random pool, and we use it as a way to \_improve\_ that random pool. So even if rdrand were to be back-doored by the NSA, our use of rdrand actually improves the quality of the random numbers you get from {\tt /dev/random}.

...
\end{quote}


\section{Intel Secure Key} \label{sec:intel-secure-key}
The Intel Secure Key (ISK) uses cascade construction, combining a HW RNG with CSPRNG into one sealed block on CPU, which is compliant with many security standards, including NIST SP800-90, FIPS-140-2, and ANSI X9.82\cite{IntelDRNGGuide}. Although it is impossible to audit it, there was found no evidence of low entropy or anything that would deny the security standards compliance - neither with tests in chapter \nameref{chap:statistical-testing}, nor any other tests anyone did\footnote{I assume any such revelation would become quickly known.}.

\section{Existing usages} 
Kernel, openssl
\section{Possible usages}

%========================================================================= 


%=========================================================================
\chapter{The library}  \label{chap:library}
According the needs of RedHat I created a library providing basic interface over the RdRand instruction as well as a simple application using this library.


\TODO{fedora package} % TODO Fedora Package


\section{API} \label{sec:library-api}
The library, if installed into the system, can be included by using {\tt \#include <rdrand-VERSION/rdrand.h>}. In the time of this work, the library is using the following API.

\subsection{Constants}
\begin{description}
  \item[RDRAND\_SUCCESS] Returned by function if a random number(s) was generated correctly.
  \item[RDRAND\_FAILURE] Returned by function if a random number(s) was NOT generated correctly.
  \item[RDRAND\_SUPPORTED] Returned by \function{rdrand_testSupport} function if the CPU support RdRand.
  \item[RDRAND\_UNSUPPORTED] Returned by \function{rdrand_testSupport} function if the CPU doesn't know RdRand.
  
\end{description}


\subsection{Functions}

\subsubsection{Non-generating functions}

These functions are not generating any random numbers.\\

\FunctionDeclare{int}{rdrand_testSupport}{void}{Detect if the CPU support RdRand instruction. Returns {\tt RDRAND_SUPPORTED}  or {\tt RDRAND_UNSUPPORTED}.}\\

\subsubsection{Simple wrappers}
These methods are simply wrappers of an ASM code which generates only one n-bits number. Although these functions are provided, I expect that they will be used only infrequently. Returns {\tt RDRAND\_SUCCESS} or {\tt RDRAND\_FAILURE}.\\

\FunctionDeclare{int}{rdrand16_step}{uint16\_t *x}{Generates 16 bits of entropy through RdRand.}\\

\FunctionDeclare{int}{rdrand32_step}{uint32\_t *x}{Generates 32 bits of entropy through RdRand.}\\

\FunctionDeclare{int}{rdrand64_step}{uint64\_t *x}{Generates 64 bits of entropy through RdRand.}\\

\subsubsection{Generating single value}
More complex functions than the previous -- in case of RdRand failure, these functions will try it again for the specified amount of times. Negative {\tt retry\_limit} implies default value with which the library is compiled. Returns {\tt RDRAND\_SUCCESS} or {\tt RDRAND\_FAILURE}.\\


\FunctionDeclare{int}{rdrand_get_uint16_retry}{uint16\_t *x, int retry\_limit}{Generates 16 bits of entropy through RdRand.}\\

\FunctionDeclare{int}{rdrand_get_uint32_retry}{uint32\_t *x, int retry\_limit}{Generates 32 bits of entropy through RdRand.}\\

\FunctionDeclare{int}{rdrand_get_uint64_retry}{uint64\_t *x, int retry\_limit}{Generates 64 bits of entropy through RdRand.}\\

\subsubsection{Generating longer values}
As a single random value is usually not enough, the library provides also functions for generating multiple bytes of random values. For higher speed, all these functions are generating values in 64bit blocks when it is possible.
These functions also accept {\tt retry\_limit} as the previous ones. Returns bytes of sucessfully generated values.\\


\FunctionDeclare{size\_t}{rdrand_get_bytes_retry}{void *dest, const size\_t size, int retry\_limit}{Generate {\tt size} bytes of random data.}\\


\FunctionDeclare{size\_t}{rdrand_get_uint64_array_retry}{void *dest, const unsigned int count, int retry\_limit}{Generate {\tt count} of 64bit blocks of random data.}\\

\FunctionDeclare{size\_t}{rdrand_get_uint32_array_retry}{void *dest, const unsigned int count, int retry\_limit}{Generate {\tt count} of 32bit blocks of random data.}\\

\FunctionDeclare{size\_t}{rdrand_get_uint16_array_retry}{void *dest, const unsigned int count, int retry\_limit}{Generate {\tt count} of 16bit blocks of random data.}\\

\FunctionDeclare{size\_t}{rdrand_get_uint8_array_retry}{void *dest, const unsigned int count, int retry\_limit}{Generate {\tt count} of 8bit blocks of random data.}\\

\FunctionDeclare{size\_t}{rdrand_fwrite}{FILE *f, const size\_t count, int retry\_limit}{Generate {\tt count} bytes of random values and write it to the {\tt f} stream}\\

\subsubsection{Secure generating}
As is documented in the chapter~\nameref{sec:rdrand-instruction}, the CPU is using an~pseudorandom generator in~connection with an~entropy source. If the~user want to avoid of the~risk of~lower entropy for some reason, it is possible to use these functions, that~guarantee by~reseeding the internal entropy pool, that each~64bit generated value is independent on the~previous or the~next one. For the~principle, also see \nameref{sec:rdrand-instruction} chapter.\\



\FunctionDeclare{size\_t}{rdrand_get_uint64_array_reseed_delay}{uint64\_t *dest, const size\_t count, int retry\_limit}{Generate {\tt count} of 64bit values. Force reseed by waiting few microseconds before each generating.}\\


\FunctionDeclare{size\_t}{rdrand_get_uint64_array_reseed_skip}{uint64\_t *dest, const size\_t count, int retry\_limit}{Generate {\tt count} of 64bit values. Force reseed by generating and throwing away 1024 values per one saved.}\\

%=========================================================================


%=========================================================================

\chapter{Rdrand-gen} \label{chap:generator}

Because the library is only for C~language, using it for example with shell scripts would be difficult. For this reason was created also a~simple executable application, which is installed with the~library, and which can be used for generating random values without the~need of using C~language.

The generator has four optional command-line parameters to~modify its behavior. Firstly, {\tt --amount} can be used to generate specific amount of~bytes of~randomness. Suffixes K, M, G and T are accepted for easier use and when this option is not used, then the~application is generating indefinitely until it is stopped, for example by KILL signal.

The second parameter is {\tt --method}, which allows user to change the default method \function{rdrand_get_bytes_retry} for the~two reseeding functions. The~names of the~methods are made shorter for the~interaction with the~user. Third parameter {\tt --output} is used for specifying the output file -- without it, the random values are printed on {\tt stdout}. 

The {\tt --threads} can specify, how many threads the~generator will run in parallel for better performance. By default it is set to 2, because according of Intel~\cite{IntelArk}, in Ivy Bridge generation of CPUs, in which the~instruction was added, there are still CPUs with just two processing units (i.e. the Celeron serie).

\TODO{What families RdRand has? All?} % TODO What families RdRand has? All?

\section{Underflow recovery}
Although is stated in Intel's Software Developer Manual~\cite{IntelSWManualVol1}, chapter 7.3.17, that an exceeding of the speed of the internal generator is unlikely, and according of {\bf unverified} information, for example on StackOverflow~\cite{StackoverflowRDRANDCharacteristics} it should not be possible in current generations (specifically on Ivy Bridge) of Intel's CPU to achieve it, we decided that the application should be working with good performance even in case of slower internal generator. The importance of this decision become even more obvious after finding that on \machine{dell-pr1700-02.lab.bos.redhat.com} the CPU wasn't able to handle more than four parallel threads reading from RdRand\footnote{Unfortunately, I cannot provide a statistic probability of such situation -- only one machine from all I have tried had this problem.}.

The principle is simple: By default, there is tolerance for few failures, implemented in the library itself, when a new call of the RdRand instruction is make immediately. But if amount of the failures in a row exceeds a limit with which the generator application was compiled, then there is a need for another approach; it is necessary to be able to work even on CPU with slow generator, yet still be able to detect complete failure of the HW.

In the case of exceeding of the HW RNG speed, the generator application tries to lower its own speed to get aligned with HW. This is at first done by decrementing threads count by one and new try. If this solution is not working or is not possible (that means, when the threads count was lowered to a single thread, or was so from the beginning), delays are being inserted between calls. The delays are then lengthened with each unsuccessful call. If even in this case the HW RNG is not able to provide enough random values, the application ends with an error message\footnote{Such situation would be clearly a sign of a hardware error and thus it is questionable if the generated values would be still really random}.

%=========================================================================


%=========================================================================

\chapter{Tests} \label{chap:tests}
For the need of performance testing I wrote another application, as well as small set of scripts for automating of tests. And after getting the code on decent performance, I also used some statistical tests batteries.

\subsection{Performance test}
Because the performance of the RNG is important, I needed to check the performance during the development on different machines and with different amount of threads. With manual testing it would be difficult to test the library in range (for example) from 1 to 120 threads.

\subsection{Scripts}
The set of Bash and Python scripts is there for automatic run of the throughput test with different count of threads, parsing the values and then creating a graph. You can see the outputs in chapter \nameref{sec:performance}.

\subsection{Statistical test}

%=========================================================================



%=========================================================================
\chapter{Performance testing} \label{chap:performance}
In RedHat I had access to machines on which I could test the library. 

\TODO{measure also times for getting one number, one KiB... Don't forget to include any referenced test!} % TODO performance tests!

\section{Half performance on some machines}

\section{Underflow}
The only machine on which I was able to achieve underflow of the HW RNG is \machine{dell-pr1700-02.lab.bos.redhat.com}.


\section{Specifications of referenced machines}
\TODO{Shouldn't be in attachments?} % TODO shouldn't be in attachments?

\machineDeclare{dell-pr1700-02.lab.bos.redhat.com}{Intel(R) Xeon(R) CPU E3-1285 v3 @ 3.60GHz}{The internal RNG is not able to handle more than four parallel threads at November 2013.}

%=========================================================================

%=========================================================================
\chapter{Statistical testing}\label{chap:statistical-testing}
Data
%=========================================================================

%=========================================================================
\chapter{Conclusion}
The End

%========================================================================= 


%=========================================================================
 % viz. obsah.tex

  % Pouzita literatura
  % ----------------------------------------------
\ifczech
  \bibliographystyle{czechiso}
\else 
  \bibliographystyle{plain}
%  \bibliographystyle{alpha}
\fi
  \begin{flushleft}
  \bibliography{bibliography} % viz. literatura.bib
  \end{flushleft}
  \appendix
  
  \chapter{Attachments}\label{chap:attachments}
\section{Content of the CD}
List of content of the attached CD.

%\chapter{Manual}
%\chapter{Konfigra�n� soubor}
%\chapter{RelaxNG Sch�ma konfigura�n�ho soboru}
%\chapter{Plakat}


 % viz. prilohy.tex
\end{document}
