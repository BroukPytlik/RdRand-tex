%=========================================================================
% (c) Michal Bidlo, Bohuslav K�ena, 2008

\chapter{Prologue}
Generally about random numbers... 

\chapter{Random numbers and deterministic machines}
How to generate them?
\section{Pseudorandom numbers}
How pseudorandom numbers can be generated.

\section{Real random numbers}
We need some entropy source - user input, network communication... and pass it as a seed to pseudorandom generator.

\chapter{The Intell's RdRand instruction}
Informations about the instruction itself - from Intell's documentation, presentations...

\chapter{My work}
What have I done?

\section{The library}
\section{API} 
The library, if installed into the system, can be included by using {\tt \#include <rdrand-VERSION/rdrand.h>}. In the time of this work, the library is using the following API.

\subsection{Constants}
\begin{description}
  \item[RDRAND\_SUCCESS] Returned by function if a random number(s) was generated correctly.
  \item[RDRAND\_FAILURE] Returned by function if a random number(s) was NOT generated correctly.
  \item[RDRAND\_SUPPORTED] Returned by \function{rdrand_testSupport} function if the CPU support RdRand.
  \item[RDRAND\_UNSUPPORTED] Returned by test-support function if the CPU doesn't know RdRand.
\end{description}


\subsection{Functions}
\FunctionDeclare{int}{rdrand_testSupport}{void}{blah blah}


\section{The generator}
Usage, failure recovery, ...

\section{Tests}
Usage, principes...

\chapter{Performance testing}
Data

\chapter{Statistical testing}
Data

\chapter{Conclusion}
The End
%=========================================================================
