%=========================================================================
\chapter{Introduction}
\TODO{Slovosled!} %TODO slovosled!
Generating of~true random numbers is a~stochastic process. In opposite, computers are deterministic machines and as such, they are unable to~generate true random numbers just using abilities of~a~Turing machine. \TODO{quote it} %TODO quote it
But random numbers are crucial in cryptography and once computers began to be used in this domain, \TODO{more details} %TODO more details
people tried combine these two conflicting requests - to~allow a~deterministic machine to act stochasticaly. 

There are two solutions of~this problem. We can stay with completely deterministic machines and through series of~mathematical operations compute pseudo random numbers, that seems to be random, but when using the~same initial state and algorithm, we get the~same numbers again. Or we can add some source of~entropy to the~system, a~device that is physically stochastic and measure its output (wherever it be temperature, radiation, or anything else what can be later, directly or indirectly, converted to electric current). The second approach is providing real random numbers, but frequently with much lower speed, than pseudo-random algorithms. Also, finding an entropy source with a~good speed, an~uniform distribution and with a~reliable price, size, energy consumption and other parameters is difficult, so frequently the~generated random values are used as a~seed for a~pseudo-random algorithm, which can lead to great speed enhancement without loosing much of~the~randomness.

The problematic of~quality of~the~entropy for a~specific purpose is wide and this work is not intended to cover it to great length, but still it has be shortly mentioned. Clearly, different requests has an developer of~a~video game, a~researcher performing a~simulation and an~army for encrypting their information. The researcher needs numbers that seems to be random, but aren't - he or she needs to be able to repeat the~simulation with the~same initial state to get the~same result\footnote{In some cases of~computing, researchers are even keeping the~same machines, as a~different machine would provide a~different results due to inner number representation and architecture differences.\cite{ArithmeticInCloud}\TODO{Is it ok?}%TODO is it ok?
}
The video game developer can also need the~repeatibility (e.g. for generating a~terrain), but in another case, like decisions of~a~artificial intelligence, the~repeatibility may not be requested and in case of~gambling highly unwanted. And the~army needs the~random numbers generator to be completely stochastic to prevent an enemy to decipher their messages. Each of~these cases has different requests for quality, speed and price.

Due to prices of~{\em Hardware Random Number Generators (HW RNG)}, they were for long time domain of~just governments and big corporations,\TODO{Quote it}%TODO quote it - security oriented CPUs with HW RNGs...
leaving the~consumer electronic to rely only on {\em Pseudo-Random Number Generators (PRNG)}. PRNG algorithms developed to great success over time, providing enough entropy for usual needs of~ordinary people and also for most of~cryptographic needs ({\em Cryptographically Secure Random Number Generators - CSPRNG}), but still it needs to be seeded by data with at least some entropy from the~beginning. And as more and more of~our daily life depends on computers, the~importance of~keeping our data secured raised up. 

In 2012\cite{IntelRdRandFindAbout} Intel added a~Digital Random Number Generator (DRNG) on their chips and allowed programmers to use it as part of~instruction set of~that CPUs. Intel named the~instruction RdRand and its own implementation and the~underlying DRNG hardware implementation is named {\em Intel Secure Key} (previously code-named Bull Mountain Technology)\cite{IntelDRNGAnalysis}. Intel used combination of~HW RNG and CSPRNG, solution known as {\em Cascade Construction RNG}, where the~relatively slow HW RNG works as a~seed generator for much faster CSPRNG. As is showed later in this thesis, in \fullref{sec:testing:performance-testing}, the~difference in speed is about thousand times.

This step brought HW RNGs to general public, but the~limitation on just Intel CPUs means that there is still big part of~the~information technology market without such solution - in x86 world there is also AMD who did not yet implemented this instruction and also many of~different architectures, like ARM, in mobile devices. So the~programmers cannot count on the~presence of~a~HW RNG on a~casual computers yet. Furthermore, the~Intel's RdRand instruction is still mostly unknown and there are also a~questions about reliability of~this generator, which is sealed into the~chip and unable to be audited\cite{TheodoreTsoNSA} if it is really manufactured according of~published scheme\cite{AnalysisOfDRNG} or if there is not a~backdoor. There is important to notice that if there could be a~backdoor in the~RNG, there could be backdoors also in any other part of~the~CPU, opening possibilities for many others attacks which could achieve the~same result.

As I'm interested in computer security on some level, as well as in Linux, when I got the~possibility to work on implementation of~a~library using RdRand, as well as test it, in production environment of~Red Hat corporation, I was interested in it and choose it as my thesis. During the~work, we have discovered unexpected issue with performance on some CPUs which was handed to Intel, yet until now without a~response.

Because now, in 2013, no other implementation of~RdRand than Intel's one exists\footnote{Although AMD is working on their own implementation.\TODO{Find more about it }},%TODO find more about it - http://extrahardware.cnews.cz/cpu-architektury-excavator-budou-umet-avx2-rdrand-dalsi-nove-instrukce
the term {\em RdRand} is used just as the~instruction implemented by {\em Intel Secure Key} technology.

\TODO{Update it when news come.}%TODO Update it when news come.


%========================================================================= 
