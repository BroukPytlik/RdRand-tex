%============================================================
\chapter{Introduction}
\par{
Generation of~true random numbers is a~stochastic process.
In opposite, computers are deterministic machines 
and thus they are unable to~generate true random numbers by using abilities 
of~a~Turing machine. 
But random numbers are crucial in cryptography and once computers began 
to be used in this domain, people tried combine these two conflicting requests 
-- to~allow a~deterministic machine to act stochasticaly. 
}

\par{
There are two solutions of~this problem. We can stay with completely 
deterministic machines and through series of~mathematical operations 
compute pseudo random numbers, that seems to be random, 
but when using the~same initial state and algorithm, we get the~same 
numbers again. Or we can add some source of~entropy to the~system, 
a~device that is physically stochastic and measure the stochastic process 
(thermal noise, radioactive decay, etc).
}

\par{
The second approach can provide real random numbers, but it requires online 
testing for case of hardware failure and also, finding of an~entropy source with
a~good speed, an~uniform distribution (without a bias) and with a~reliable price,
size, energy consumption and other parameters is difficult. Because of this, 
a lower quality HW generator is connected with an device that tests the bias 
and selects only some bits and then the~generated random values are used 
as a~seed for a~crypthographicaly secure pseudo-random generator, 
which can lead to great speed enhancement without loosing much 
of~the~randomness.
}

\par{
The problematic of~quality of~the~entropy for a~specific purpose is wide 
and this work is not intended to cover it to great length, but still this area 
has to be shortly mentioned. Clearly, different requests has an developer 
of~a~video game, a~researcher performing a~simulation and an~army 
for encrypting their information. 
The researcher needs numbers that seems to be random, but aren't 
-- he or she needs to be able to repeat the~simulation with the~same initial state 
to get the~same result\footnote{In some cases of~computing, 
researchers are even keeping the~same machines, as a~different machine 
would provide a~different results due to inner number representation and architectural differences.\cite{ArithmeticInCloud}}.
}

\par{
The video game developer can also need the~repeatibility (e.g. for generating 
a~terrain), but in another case, like decisions of~a~artificial intelligence, 
the~repeatibility may not be requested and in case of~gambling highly unwanted. 
And the~army needs the~random numbers generator to be completely stochastic 
to prevent an enemy to decipher their messages. Another example, where random 
numbers are used, is the {\em Monte Carlo} method of solving definite integrals. 
Each of~these cases has different requests for quality, speed and price.
}

\par{
Due to prices of~{\em Hardware Random Number Generators (HW RNG)} 
and because the few cheap solutions never got widely used, 
they were for long time domain of~just governments and big corporations,
leaving the~consumer electronic to rely only on {\em Pseudo-Random Number 
Generators (PRNG)}. PRNG algorithms developed to great success over time, 
providing enough entropy for usual needs of~ordinary people and also for most 
of~cryptographic needs ({\em Cryptographically Secure Random Number 
Generators - CSPRNG}), but still it needs to be seeded by data with at least 
some entropy from the~beginning. And as more and more of~our daily life 
depends on computers, the~importance of~keeping our data secured have 
raised up. 
}

\par{
In 2012\cite{IntelRdRandFindAbout} Intel added a~Digital Random Number 
Generator (DRNG) on their chips in Ivy Bridge generation
and allowed programmers to use it as part 
of~instruction set of~that CPUs. Intel named the~instruction RdRand and its own 
implementation and the~underlying DRNG hardware implementation is named 
{\em Intel Secure Key} (previously code-named Bull Mountain Technology)
\cite{IntelDRNGAnalysis}. Intel used combination of~HW RNG and CSPRNG, 
solution known as {\em Cascade Construction RNG}, where the~relatively slow 
HW RNG\footnote{The HW RNG itself has output about 3 gigabits per second
\cite{BehindRdRand}, but these values are biased, so amount of the bites 
is reduced to concentrate the randomness.} works as a~seed generator 
for much faster CSPRNG. As is showed later in this thesis, 
in \fullref{sec:testing:performance-testing}, the~difference in speed 
is about thousand times.
}

\par{
This step brought HW RNGs to general public, but the~limitation on only Intel 
CPUs means that there is still big part of~the~information technology market 
without such solution - in x86 world there is another big player, 
AMD, who did not yet implemented this instruction and also many 
of~different architectures, like ARM, in mobile devices. 
So the~programmers cannot count on the~presence of~a~HW RNG 
on a~casual computers yet. 
}

\par{
Furthermore, the~Intel's RdRand instruction is still mostly unknown 
and there are also a~questions about reliability of~this generator, 
which is sealed into the~chip and unable to be audited\cite{TheodoreTsoNSA} 
if it is really manufactured according of~published 
scheme\cite{AnalysisOfDRNG} or if there is a~backdoor. 
}

\par{
There is important to notice that if there could be a~backdoor in the~RNG, 
there could be possibly backdoors also in any other part of~the~CPU, opening 
possibilities for many others attacks which could achieve {\em the~same} result. 
But hiding a backdoor to RdRand could be done more easily than to, 
for example, a HW acceleration for AES encrypting, so I agree with Linus 
Torvalds that RdRand alone is great if we do not need crypthographicaly secure 
numbers, but for cryptographic usage, it is better to use it just as one of more 
sources of entropy and seed some CSPRNG by it.
}

\par{
As I'm interested in computer security (on some level), 
as well as I'm interested in Linux, 
when I got the~possibility to work on implementation of~a~library using RdRand
in production environment of~Red Hat corporation, 
I was interested in it and choose it as my thesis. 
During the~work, we have discovered unexpected issue with half performance 
on some CPU. This was handed to Intel, yet without a result. 
}
\par{
Because now, in spring 2014, no other implementation of~RdRand than Intel's 
one exists\footnote{Although AMD is working on their own implementation 
for their future Excavator architecture~\cite{AMDRdRand}, probably named 
RDRND.}
the term {\em RdRand} is used just as the~instruction implemented 
by {\em Intel Secure Key} technology.
}


%============================================================
