%=========================================================================

\chapter{Rdrand-Gen} \label{chap:generator}
\par{
Because the~library is only for C~language, using it for example with shell scripts would be difficult. For this reason, a~simple executable application was also created, which is installed with the~library, and~which can be used for generating random values without the~need of~using C~language by~the~user.
}

\par{
The~generator has four optional command-line parameters to~modify its behavior. Firstly, {\tt --amount} can be used to~generate specific amount of~bytes of~randomness. Suffixes K, M, G and~T are accepted for easier use and~when this option is not used, then the~application is generating indefinitely until it is stopped, for example by~KILL signal.
}

\par{
The~second parameter is {\tt --method}, which allows user to~change the~default method \\\function{rdrand_get_bytes_retry} for the~two reseeding functions. The~names of~the~methods are made shorter for the~interaction with the~user. Third parameter {\tt --output} is used for specifying the~output file -- without it, the~random values are printed on {\tt stdout}. 
}

\par{
The~{\tt --threads} can specify, how many threads the~generator will run in parallel for better performance. By~default it is set to~2, because according to~Intel~\cite{IntelArk}, in Ivy Bridge generation of~CPUs (in which the~instruction was added) there are still CPUs with just two processing units (i.e. the~Celeron series).
}

\section{Underflow Recovery}
\par{
Although is stated in Intel's Software Developer Manual~\cite[chapter~7.3.17]{IntelSWManualVol1} that an exceeding of~the~speed of~the~internal generator is unlikely\footnote{In case of failure during self-tests, the RdRand call will always fail.}, and~although according to~{\em unverified} information (for example on StackOverflow~\cite{StackoverflowRDRANDCharacteristics}) it should not be possible in current generations (specifically on Ivy Bridge) of~Intel's CPU to~achieve it, we decided that the~application should be working with good performance even in case of~slower internal generator. The~importance of~this decision become even more obvious after finding that on \machine{dell-pr1700-02.lab.bos.redhat.com} the~CPU wasn't able to~handle more than four parallel threads reading from RdRand\footnote{Unfortunately, I cannot provide a~statistic probability of~such situation -- only one machine from all I have tried had this problem and~I wasn't able to~find that anyone other would experienced this too, which is, with regards of~the~currently low usage of~the~instruction, not surprising.}.
}

\par{
The~principle is simple: By~default, there is tolerance for few failures, implemented in the~library itself. In such case, a~new call of~the~RdRand instruction is made immediately after a~failure. But if the~RdRand is too slow, amount of~the~failures in a~row exceeds a~specific limit\footnote{Currently 3, but can be changed in the~source code.}. 
}

\par{
In case of~exceeding of~the~HW RNG speed, the~generator application tries to~lower the~speed of~acquiring. This is at first done by~decrementing threads count by~one and~new try. If this solution is not working or is not possible (that means, when the~threads count was lowered to~a~single thread, or was such from the~beginning), delays are being inserted between calls. The~delays are then lengthened with each unsuccessful call. If even in this case the~HW RNG is not able to~provide enough random values, the~application ends with an error message\footnote{Such situation would be clearly a~sign of~a~hardware error and~thus it is questionable if the~generated values would be still really random}.
}
%=========================================================================
